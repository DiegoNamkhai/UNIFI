# Prodotti notevoli
| OPERAZIONE | FORMULA |
|---|---|
|quadrato di un trinomio | $$(a+b+c)^3= a^2 + b^2 + c^2 + 2ab + 2ac + 2bc$$ |
|particolari prodotti notevoli | $$(a+b)(a^2-ab+b^2)=a^3+b^3$$ $$(a-b)(a^2+ab+b^2)=a^3-b^3$$|

# Operazioni con i radicali
| OPERAZIONE | FORMULA |
|---|---|
|semplificazione | $$\sqrt[mn]{a^n} = \sqrt[m]{a}$$ |
|potenza radicale | $$(\sqrt[n]{a})^m = \sqrt[n]{a^m}$$ |
|razionalizzazione | $$\frac{b}{\sqrt{a}}=\frac{b}{\sqrt{a}} \cdot \frac{\sqrt{a}}{a} = \frac{b \sqrt{a}}{a}$$ $$\frac{b}{\sqrt[n]{a^m}}=\frac{b}{\sqrt[n]{a^m}} \cdot \frac{\sqrt[n]{a^{n-m}}}{\sqrt[n]{a^{n-m}}} = \frac{b \sqrt[n]{a^{n-m}}}{a}$$ $$\frac{Q}{\sqrt{a} + \sqrt{b}}=\frac{Q}{\sqrt{a} + \sqrt{b}} \cdot \frac{\sqrt{a} - \sqrt{b}}{\sqrt{a} - \sqrt{b}} = \frac{Q \sqrt{a} - \sqrt{b}}{a-b}$$ $$\frac{Q}{\sqrt{a} - \sqrt{b}}=\frac{Q}{\sqrt{a} - \sqrt{b}} \cdot \frac{\sqrt{a} + \sqrt{b}}{\sqrt{a} + \sqrt{b}} = \frac{Q \sqrt{a} + \sqrt{b}}{a-b}$$ |
|potenza radicale | $$(\sqrt[n]{a})^m = \sqrt[n]{a^m}$$ |

# Limiti notevoli
| FUNZIONE | LIMITE |
|---|---|
|logaritmo naturale | $$\lim_{x\to\ 0} \frac{ln(1+x)}{x}=1$$ |
|funzione logaritmica | $$\lim_{x\to\ 0} \frac{log_a{(1+x)}}{x}= \frac{1}{ln(a)}$$ |
|funzione esponenziale | $$\lim_{x\to\ 0} \frac{e^x -1}{x}=1$$ |
|funzione esponenziale con base arbitraria | $$\lim_{x\to\ 0} \frac{a^x -1}{x}=ln(a)$$ $$con \quad a>0$$|
|numero di Nepero | $$\lim_{x\to\infty} \bigg(1+ \frac{1}{x} \bigg)^x=e$$ |
|potenza con differenza | $$\lim_{x\to\ 0} \frac{(1+x)^c -1}{x}=c$$ $$con \; c \in \mathbb{R} $$ |
|funzione sin | $$\lim_{x\to\ 0} \frac{sin(x)}{x}=1$$ |
|funzione cos | $$\lim_{x\to\ 0} \frac{1-cos(x)}{x^2}= \frac{1}{2}$$ |
|funzione tan | $$\lim_{x\to\ 0} \frac{tan(x)}{x}=1$$ |
|arcsin | $$\lim_{x\to\ 0} \frac{arcsin(x)}{x}=1$$ |
|arctan | $$\lim_{x\to\ 0} \frac{arctan(x)}{x}=1$$ |
|sin parabolico | $$\lim_{x\to\ 0} \frac{sinh(x)}{x}=1$$ |
|cos parabolico | $$\lim_{x\to\ 0} \frac{cosh(x)-1}{x^2}= \frac{1}{2}$$ |
|tan parabolico | $$\lim_{x\to\ 0} \frac{tanh(x)}{x}=1$$ |

# Derivate fondamentali
 |Derivata| f(x) | f'(x) |
|---|---|---|
|costante|$$f(x) = costante$$  | $$f'(x)=0$$ |
|x|$$f(x) = x$$  | $$f'(x)=1$$ |
|potenza|$$f(x) = x^s, \; s \in \mathbb{R}$$ | $$f'(x)=sx^{s-1}$$ |
|esponenziale|$$f(x) = x$$  | $$f'(x)=1$$ |
|$e^x$|$$f(x) = e^x$$  | $$f'(x)=e^x$$ |
|logaritmo|$$f(x) = log_a{x}$$  | $$f'(x)= \frac{1}{x \ ln(a)}$$ |
|sin|$$f(x) = sin(x)$$  | $$f'(x)=cos(x)$$ |
|cos|$$f(x) = cos(x)$$  | $$f'(x)=-sin(x)$$ |
|tan|$$f(x) = tan(x)$$  | $$f'(x)= \frac{1}{cos^2(x)}$$ |
|cot|$$f(x) = cot(x)$$  | $$f'(x)= - \frac{1}{sin^2(x)}$$ |
|arcsin|$$f(x) = arcsin(x)$$  | $$f'(x)= \frac{1}{ \sqrt{1-x^2}}$$ |
|arccos|$$f(x) = arccos(x)$$  | $$f'(x)= - \frac{1}{ \sqrt{1-x^2}}$$ |
|arctan|$$f(x) = arctan(x)$$  | $$f'(x)= \frac{1}{1+x^2}$$ |
|arccot|$$f(x) = arccot(x)$$  | $$f'(x)= - \frac{1}{1+x^2}$$ |
|sinh|$$f(x) = sinh(x) = \frac{e^x-e^{-x}}{2}$$  | $$f'(x)= cosh(x)$$ |
|cosh|$$f(x) = cosh(x) = \frac{e^x+e^{-x}}{2}$$  | $$f'(x)= sinh(x)$$ |

















